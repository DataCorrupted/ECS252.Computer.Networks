
\section*{Problem 3}

In this problem we will work through the idea of statistical multiplexing that we discussed in class.
Nodes are connected to a switch using high-bandwidth links, i.e., the links between the nodes and the switch are not
limiting.
The outgoing link form the switch is $M$ Mbps.
Consider that each node is active with probability $p$ and when it is active it transmits data at $R$ Kbps.
We define congestion when the instantaneous rate is greater than the link rate.
Write a code (Matlab, R, or Python) to answer the following questions.

\begin{enumerate}
      \item Assume that $M = 1$ Mbps, $R = 200$ Kbps, and $p = 0.2$.
            Plot the probability of congestion $P_c$ as a
            function of $N$.
      \item Repeat the previous case with $M = 1$ Mbps, $R = 800$ Kbps, and $p = 0.01$.
      \item Here we turn the problem around to answer a capacity estimation problem.
            We are given that $N = 20$, $R = 200$ Kbps, and $p = 0.2$.
            We are also given that the allowed probability of congestion $P_c$ is $0.1$.
            Rework the code in part (1) to find the minimum value of $M$ that is needed.
\end{enumerate}

\subsection*{Solution}

Let's denote $N_M = \lfloor M / R \rfloor$.
$N_M$ describes the critical number of devices in the network below which there will never be a congestion.
\[
      P_c(N) = \left\{ \begin{array}{ll}
            0                                                      & N \le N_M \\
            1 - \sum_{k=0}^{N_M} {{N} \choose {k}} p^k (1-p)^{N-k} & N > N_M   \\
      \end{array}
      \right.
\]

\begin{figure}[h]
      \centering
      \begin{tikzpicture}
            \begin{axis}[
                        title={$P_C(N)$ given $M=1000, R=200, p=0.2$},
                        xlabel={$N$},
                        ylabel={$P_c(N)$},
                        legend pos=south east,
                        ymajorgrids=true,
                        grid style=dashed,
                  ]
                  \addplot[red] table [x=N, y=Pc, col sep=comma] {plots/Q3.Pc_N_1000_200_0.2.csv};
            \end{axis}
      \end{tikzpicture}
\end{figure}
\begin{figure}[h]
      \centering
      \begin{tikzpicture}
            \begin{axis}[
                        title={$P_C(N)$ given $M=1000, R=800, p=0.01$},
                        xlabel={$N$},
                        ylabel={$P_c(N)$},
                        legend pos=south east,
                        ymajorgrids=true,
                        grid style=dashed,
                  ]
                  \addplot[red] table [x=N, y=Pc, col sep=comma] {plots/Q3.Pc_N_1000_800_0.01.csv};
            \end{axis}
      \end{tikzpicture}
\end{figure}

Using binary search we can easily get $M$ should at least be 1199 Kbps

The code that is used to generated the graph and the binary search is attached:

\begin{minipage}{\linewidth}
      \lstinputlisting[language=Python]{src/q3.py}
\end{minipage}