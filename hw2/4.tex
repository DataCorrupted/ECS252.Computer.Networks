
\section*{Problem 4}

Read the paper titled ``Incentives Build Robustness in BitTorrent'' and write a short summary.
Briefly describe the architecture and key algorithms that are used.
You will find this paper in the Reference tab in Canvas.

\subsection*{Solution}

BitTorrent solves the download threshold problem.
While the host server can host a file, the more the downloaders are, the slower their bandwidth would be.
To solve this, one can use P2P sharing.
However, there are serval challenges that BitTorrent solves.

\begin{enumerate}
    \item Peer location. It's not known to any peer where other peers are(in terms of address)
    \item File partitioning. It's hard to figure out for each peer who has which part.
    \item Time partitioning. Peers are not like 247 servers, they are online for little time in an un-controllable manner.
    \item Fairness. How to prevent peers from downloading without sharing?
\end{enumerate}

BitTorrent uses a meta file ending with \textit{.torrent}(hereby referenced as torrent), which stores the metadata of the files that's been shared.
The metadata includes piece size, number and the hash of each piece; most importantly, there is one or more urls pointing to a tracker server.
Tracker server tracks all peers downloading torrent, including address, port, etc.
Once a peers contacts tracker, a list of peers are returned and the peer can select anyone to contact.
A pipeline is also setup, a piece is cut to sub-pieces and every time a sub-piece is  arrived a new request is sent, thus higher the transfer rate.
Lastly, piece selection helps improves the systems throughput.
There are 4 policies available:

\begin{enumerate}
    \item Strict policy, select piece by piece in the order.
    \item Rarest first, so that there are always enough copy in case one peer goes down.
    \item Random piece, before a new comer can start uploading, random pieces are selected.
    \item Endgame, a chocking but slow piece will be request early on so that it won't delay the overall finishing time.
\end{enumerate}

The last remaining problem is chocking, i.e. a peer may decide not to upload.
BitTorrent uses a simple solution: any peer would prefer upload to peers that just uploaded to them(i.e. the peers they just downloaded from.)
Thus, peers would be inclined to share.
However, each peer always chock 4 peers to prevent congestion, decision is made based on download rate.
The decidion is made every 10 seconds to prevent frequent chocking/unchocking.
A new comer noting to upload, thus everyone chocks them using previous algorithm.
To prevent this from happening, every peer unchocks a peer for every 30 seconds, the justification is that this time would be suffice to download a piece and get going.
Finally, it is possible one peer is chocked by everyone he is downloading from.
BitTorrent decides it's been ``snubbed'' after one minute without a piece, it would stop uploading to them and pick other peers.
