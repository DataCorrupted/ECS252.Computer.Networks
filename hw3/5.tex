
\section*{Problem 5}

Suppose Host A and B are directly connected with a 1 Gbps link.
There is one TCP connection betweenthe two hosts, and Host A is sending to Host B an enormous file over this connection.
Host A can send its application data into its TCP socket at a rate as high as 120 Mbps but host B can read out of its TCP bufferat a maximum rate of 50 Mbps.
Describe the effect of TCP flow control.

\subsection*{Solution}

The detailed behavior differs depending on the congestion control algorithm used.
Therefore, we only discuss the general behavior here.

Initially, A will start with small bandwidth and increase that bandwidth until 120Mbps.
B can keep up with the speed until 50Mbps threshold is reached.
Then congestion control kicks in and A will lower it's sending speed.
In the end(suppose the file is not finished transmitting by then), A can send at approximately 50Mbps.