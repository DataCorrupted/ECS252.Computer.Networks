\section*{Problem 1}

Consider an end-to-end TCP connection in which the last hop is a wireless link.
Assume that MSS is 1500 bytes.
\begin{enumerate}
      \item  Suppose that packet error rate in the wireless link is $10^{-3}$.
            If the RTT is 50 msec, what is expected throughput of the TCP connection?
            (See Jupyter Notebook for an analysis of TCP throughput underrandom losses.)
      \item  What is the disadvantage of the TCP congestion control in the above scenario?
      \item  Suppose we can leverage a link layer error correction feature that can reduce the packet error rate to $10^{-5}$.
            However, it increases the RTT by 50 msec.
            What will be the expected throughput of the TCP connection in this case?
\end{enumerate}

\begin{enumerate}
      \item Using the analysis in hte notebook, we have $L = 10^{-3}$, $\text{MSS} = 1500 \text{bytes}$, $\text{RTT} = 50 \text{ms}$
            Thus we have:
            $$ \nu = \frac{1.22 * \text{MSS}}{\text{RTT} \sqrt{L}} = \frac{1.22 * 1500 * 8}{.05 * \sqrt{10^{-3}}} \text{bps} \approx 9.259 \text{Mbps}$$

      \item Congestion control kicked in when there is no congestion, slows down the whole link.

            Congestion control algorithm is based on the assumption that packet loss is rare, thus used packet loss to directly indicate a congestion.
            However, this asusmption is not true in wireless connection when packet loss is more often than cables.
            This false positive slowed the throughput of the whole link.

      \item Using the analysis in hte notebook, we have $L = 10^{-5}$, $\text{MSS} = 1500 \text{bytes}$, $\text{RTT} = 100 \text{ms}$
            Thus we have:
            $$ \nu = \frac{1.22 * \text{MSS}}{\text{RTT} \sqrt{L}} = \frac{1.22 * 1500 * 8}{.1 * \sqrt{10^{-5}}} \text{bps} \approx 46.296 \text{Mbps}$$

\end{enumerate}