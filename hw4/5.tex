\section*{Problem 5}

In this problem you will try to derive the efficiency of a CSMA/CD-like multiple access protocol.
We will consider a slotted system in which all the adapters are synchronized to the slot.
The length of the slot (in seconds) is much less than the time to transmit a frame.
Let $S$ be the length of a slot and all frames are of constant length $L=kRS$, where $R$ is the transmission rate of the channel and $k$ is a large integer.
Suppose there are $N$ nodes each with an infinite number of frames to send.
We also assume that $d_{prop} < S$, so that all nodes can detect collision before the end of a slot time.
The protocol is as follows:

\begin{itemize}
      \item If for a given slot, no node has possession of the channel, all nodes contend for the channel; inparticular, each node transmits in the slot with probability $p$.
            If exactly one node transmits in the slot, that node takes possession of the channel for the subsequent $k-1$ slots and transmits its entire frame.
      \item If some node has possession of the channel, all other nodes refrain from transmitting until the node that possesses the channel has finished transmitting its frame.
\end{itemize}

Once this node has transmitted its frame, all node contend for the channel.
Note that the channel alternates between two states: the productive state, which last exactly $k$ slots, and the non-productive state which lasts for a random number of number of slots.
The channel efficiency is the ratio of $k/(k+x)$ where $x$ is the expected number of consecutive unproductive slots.

\begin{enumerate}
      \item For a fixed $N$ and $p$, determine the efficiency of this protocol.
      \item For a fixed $N$, determine $p$ that maximizes the efficiency.
      \item Using the $p$(which is a function of N)found in (b), determine the efficiency as $N$ approaches infinity.
      \item Show that this efficiency approaches 1 as the frame length becomes large.
\end{enumerate}

\subsection*{Solution}

\begin{enumerate}
      \item The probability of successfully get the channel:
            $$P = {N \choose 1}p(1-p)^{N-1}$$
            The probability function of the number of consecutive unproductive:
            $$f(x) = (1-P)^{x-1}P$$
            Thus the expected waiting unproductive slots is:
            $$E(X) = \sum_s^N sf(s) = \frac{P}{1-P}\sum_s^N s(1-P)^s = \frac{P}{1-P}\frac{1-P}{P^2} = 1/P $$
            Thus the efficiency is:
            $$ \frac{k}{k+x} = \frac{k}{k+1/P-1} = \frac{Npk(1-p)^{N-1}}{Np(k-1)(1-p)^{N-1} + 1}$$

      \item To maximize the efficiency, is to minimize $1/P$, i.e. maximize $P$.
            For fixed $N$ we have: $$P(p) = Np(1-p)^{N-1}, p \in (0, 1)$$
            $\log$ is a monotone function, thus taking log wouldn't affect the maximuze point of this function, but taking derivitive would be much easier:
            $$L(p) = \log P(p) = \log N + \log p + (N-1)\log (1-p), p \in (0, 1)$$
            $$L'(p) = \frac{1}{p} - \frac{N-1}{1-p}$$
            The zero point of $L'(p) = 0$ is $p = \frac{1}{N}$.
            Therefore when $p \in (0, \frac{1}{N}], L'(p) > 0, L(p)$ increases, $P(p)$ increases.
            When $p \in [\frac{1}{N}, 1), L'(p) < 0, L(p)$ decreases, $P(p)$ decreases.

            Thus $P(p)$ is maximuzed when $p = \frac{1}{N}$, $P(\frac{1}{N}) = (\frac{N-1}{N})^{N-1}$

      \item Using results in 1) and 2), when $p = \frac{1}{N}$, we have:
            $$\lim_{N \rightarrow \infty} \frac{1}{P} = \lim_{N \rightarrow \infty} (\frac{N}{N-1})^{N-1} \lim_{N \rightarrow \infty} (1 + \frac{1}{N-1})^{N-1}= e$$
            Thus $$\lim_{N \rightarrow \infty} \frac{k}{k + 1/P-1} = \frac{k}{k+e-1}$$

      \item Frame length become large implies that $k \rightarrow \infty$.
            $1/P$ is a finite number, thus the efficiency can be expressed as:
            $$ \lim_{k \rightarrow \infty} \frac{k}{k + 1/P - 1} = 1$$
\end{enumerate}