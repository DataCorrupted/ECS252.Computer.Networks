\documentclass{article}

\title{ECS 252: Computer Networks \\ Project 1}
\author{Yuyang(Peter) Rong \\917781535 \\ PtrRong@ucdavis.edu}

\usepackage[utf8]{inputenc}
\usepackage{graphicx}
\usepackage[colorlinks,linkcolor=red]{hyperref}
\usepackage{amsmath, amsthm, amssymb}
\usepackage[shortlabels]{enumitem}
\usepackage{subfloat}
\usepackage{booktabs}
\usepackage{color}
\definecolor{mygreen}{rgb}{0,0.6,0}
\definecolor{mygray}{rgb}{0.5,0.5,0.5}
\definecolor{mymauve}{rgb}{0.58,0,0.82}

% For listings
% In case we need rust format, check https://github.com/denki/listings-rust
\usepackage{listings}
\lstset{ %
    frame=single,
    % numbers=left,
    backgroundcolor=\color{white},   % choose the background color
    basicstyle=\ttfamily\footnotesize,        % size of fonts used for the code
    breaklines=true,                 % automatic line breaking only at whitespace
    captionpos=b,                    % sets the caption-position to bottom
    commentstyle=\color{mygreen},    % comment style
    escapeinside={(*@}{@*)},         % if you want to add LaTeX within your code
    keywordstyle=\color{blue},       % keyword style
    stringstyle=\color{mymauve},     % string literal style
    tabsize=1,
    % where to put the line-numbers; possible values are (none, left, right)
    numbers = left,
    % how far the line-numbers are from the code
    numbersep = 10 pt,
    % the style that is used for the line-numbers
    numberstyle = \ttfamily,
    % the step between two line-numbers. If it's 1, each line will be numbered 
    stepnumber = 1,
}
\usepackage{ulem}
\usepackage{mathtools}

% For autoref & section reference.
\usepackage{hyperref}
\def\sectionautorefname{Section}
\def\subsectionautorefname{Section}
\def\subsubsectionautorefname{Section}


\usepackage{tikz}
\usepackage{float}
\usepackage{pgfplots}
\pgfplotsset{
    xticklabel style={
        /pgf/number format/fixed,
        /pgf/number format/precision=1
    }}


\usepackage{subfigure}

\begin{document}
\maketitle

\section{Overview}

In this project you will simulate and analyze the basic ALOHA protocol and investigate some of the proper-ties of the binary exponential backoff algorithm of the IEEE 802.3 Ethernet protocol.
Before you get startedyou should read Random Access Protocols Section 6.3.2 of the text.
We will cover it in class but you shouldread ahead.

In order to develop the simulation model, we will make the following assumptions:

\begin{enumerate}
    \item We will assume that time is slotted into equal length of time slots.
          In the subsequent discussion, thelength of the time slot will be denoted by $T_s$.
    \item We will let $N$ denote the number of hosts.
          We will assume the hosts are identical and packets arriveto each host following a Poisson process with rate $\lambda$(pkts/sec).
    \item Hosts can transmit only at slot boundaries.
    \item For a new packet, the node attempts transmission in the next slot boundary.
    \item If at a particular slot boundary there are more than one host ready to transmit, there will be a collision.When there is a collision the packet is not received by the receiver and must be re-transmitted.
    \item How a node determines when to re-transmit depends on the algorithm.
          We will consider differentalgorithms as discussed in the next section.
    \item We will be plotting the throughput as a function of the arrival rate $\lambda$.
          Throughput is defined as the number of successful transmission per time unit.
          In the simulation, you can count the number of slots in which there is successful transmission and divided that by the total number of slots that you simulate.
\end{enumerate}

\section{Algorithms}


We will consider the following algorithms:

\begin{enumerate}
    \item p-Persistent ALOHA: In this case each active node re-transmits in a slot with probability $p$.
          We will consider two different values of $p$, 1)$p= 0.5$ and 2)$p=1/N$.
    \item Binary Exponential Backoff: When hosts collide, they will schedule their re-transmission usingthe following binary exponential backoff algorithm.
          The number of slots to delay after then the re-transmission attempt is chosen as a uniformly distributed integer in the range $0 \le r \le 2^K$, where $K= min(n,10)$.
    \item Linear Backoff: When hosts collide, they will schedule their re-transmission using the following linear backoff algorithm.
          The number of slots to delay after then the re-transmission attempt is chosen as a uniformly distributed integer in the range $0 \le r \le K$, where $K= min(n,1024)$.
\end{enumerate}

\section{Simulation Analysis}
Extend the simulation model of the single server queue to model the above system.
Based on the simulation model, obtain the following results:

\begin{enumerate}
    \item Plot  the  throughput  as  a  function  of $\lambda$ with  the p−Persistent  Aloha,  Binary  Exponential  Backoff algorithm as described above.
          Slot time $T_s= 1$ and number of hosts $N = 30$.
          Obtain the throughputfor ten values of $\lambda$ in the range $[0.003, 0.03]$.
          Write a short discussion about the results.
    \item Here are the requirements for submitting a executable version of the code.
          Use the the following tagsfor the four different algorithms ``pp'', ``op'', ``beb'', and ``lb'' for 0.5 persistent, 1/N persistent, binary exponential backoff, and linear backoff, respectively.
          \begin{enumerate}
              \item Name the code: ethernet-simulation.py.
              \item It should take 2 parameters: a) the algorithm and b) the arrival rate
              \item It should output the throughput to 2 decimal places
          \end{enumerate}

\end{enumerate}


\subsection*{Solution}

\begin{figure}[H]
    \centering
    \begin{tikzpicture}
        \begin{axis}[
                title={Throughput when p-Persistent($p = .5$) is used},
                %xmin=0, xmax=0.033,
                %xtick={0, 0.003, ..., 0.033},
                ymode=log,
                xlabel={$\lambda$},
                ylabel={Throughput},
                legend pos=north east,
                ymajorgrids=true,
                grid style=dashed,
            ]
            \addplot[red] table [x=lambda, y=throughput, col sep=comma] {plots/PPersistentBackOff(p=0.500).csv};
            \legend{p-Persistent($p = .5$)}
        \end{axis}
    \end{tikzpicture}
\end{figure}

It can be seen that p-Persistent($p = .5$) has the worth throughput.
The best throughput is reached $\approx  0.05$, when $\lambda = 0.003$.
It's not hard to image that once we have more than 2 stations trying to colliding at the same time, the probability the next time they would collide is still high.
Suppose $k$ stations collided at a time slot, $t$ is the number of the stations deciding to re-transmit in the next time slot, the probability that next time there will be only one station trying to re-transmit is $P(t=1) = k2^{-k}$.
It's also easy to tell that $P(t=0) = 2^{-k}$.
Thus, the next time the collision rate is $P(t>1) = 1 - (k+1)2^{-k}$.
$P(t>1) \ge 0.5$ when $k \ge 3$, meaning that once we have 3 or more stations trying to re-transmit, it would be hard to get rid of this congestions.

\begin{figure}[H]
    \centering
    \begin{tikzpicture}
        \begin{axis}[
                title={Throughput for p-Persistent($p = 1/30$), Binary Exponential Backoff and Linear Backoff},
                %xmin=0, xmax=0.033,
                %xtick={0, 0.003, ..., 0.033},
                ymin=0, ymax=0.8,
                %ytick={0, 0.1, ..., 0.7},
                xlabel={$\lambda$},
                ylabel={Throughput},
                legend pos=outer north east,
                ymajorgrids=true,
                grid style=dashed,
            ]
            \addplot[red] table [x=lambda, y=throughput, col sep=comma] {plots/PPersistentBackOff(p=0.033).csv};
            \addplot[blue] table [x=lambda, y=throughput, col sep=comma] {plots/LinearBackOff.csv};
            \addplot[green] table [x=lambda, y=throughput, col sep=comma] {plots/BinaryExpBackOff.csv};
            \legend{$p = 1/30$-Persistent, Binary Exponential, Linear Backoff}
        \end{axis}
    \end{tikzpicture}
\end{figure}

In the lone term, linear backoff is the second worst algorithm.
The throughput peaked at $\approx 0.20$ and dropped to $\approx 0.15$.
The peak can be explained by higher collision rate when the packet number rises.

We can find that the max throughput for p-Persistent($p = 1/30$) is reached when $\lambda \approx 0.01$.
When $\lambda \le 0.01$, the throughput is rising with $\lambda$, which indicates that the system is not in the maximum payload.
When $\lambda \ge 0.01$, the throughput can be maximized to $\approx 0.37$
Using similar analysis we did above, we can find that even $p = 1/30$ did fairly good compared to $p=0.5$, but when the number of colliding stations gets high, it's still hard to deal with.

Binary exponential backoff have the best throughput.
The throughput keeps rising with the $\lambda$, peaking at $\approx 0.75$ when $\lambda = 0.04$.
This should be the max throughput since the packets input($30 * \lambda = 1.2$) is higher than the packets output($1$).
The outstanding throughput compared to linear backoff can be contributed to exponential growth of the waiting time.
This made sure that we only need a few collisions before we can send out a packet.
Instead, for linear backoff algorithm, the number of collisions needed is linear and thus more collisions are presented.
\end{document}
\grid

