\documentclass{article}

\title{ECS 252: Computer Networks \\ Project 1}
\author{Yuyang(Peter) Rong \\917781535 \\ PtrRong@ucdavis.edu}

\usepackage[utf8]{inputenc}
\usepackage{graphicx}
\usepackage[colorlinks,linkcolor=red]{hyperref}
\usepackage{amsmath, amsthm, amssymb}
\usepackage[shortlabels]{enumitem}
\usepackage{subfloat}
\usepackage{booktabs}
\usepackage{color}
\definecolor{mygreen}{rgb}{0,0.6,0}
\definecolor{mygray}{rgb}{0.5,0.5,0.5}
\definecolor{mymauve}{rgb}{0.58,0,0.82}

% For listings
% In case we need rust format, check https://github.com/denki/listings-rust
\usepackage{listings}
\lstset{ %
    frame=single,
    % numbers=left,
    backgroundcolor=\color{white},   % choose the background color
    basicstyle=\ttfamily\footnotesize,        % size of fonts used for the code
    breaklines=true,                 % automatic line breaking only at whitespace
    captionpos=b,                    % sets the caption-position to bottom
    commentstyle=\color{mygreen},    % comment style
    escapeinside={(*@}{@*)},         % if you want to add LaTeX within your code
    keywordstyle=\color{blue},       % keyword style
    stringstyle=\color{mymauve},     % string literal style
    tabsize=1,
    % where to put the line-numbers; possible values are (none, left, right)
    numbers = left,
    % how far the line-numbers are from the code
    numbersep = 10 pt,
    % the style that is used for the line-numbers
    numberstyle = \ttfamily,
    % the step between two line-numbers. If it's 1, each line will be numbered 
    stepnumber = 1,
}
\usepackage{ulem}
\usepackage{mathtools}

% For autoref & section reference.
\usepackage{hyperref}
\def\sectionautorefname{Section}
\def\subsectionautorefname{Section}
\def\subsubsectionautorefname{Section}


\usepackage{tikz}
\usepackage{pgfplots}
\pgfplotsset{scaled x ticks=false}

\usepackage{subfigure}

\begin{document}
\maketitle

\section{Overview}

In this project you will simulate and analyze the basic ALOHA protocol and investigate some of the proper-ties of the binary exponential backoff algorithm of the IEEE 802.3 Ethernet protocol.
Before you get startedyou should read Random Access Protocols Section 6.3.2 of the text.
We will cover it in class but you shouldread ahead.

In order to develop the simulation model, we will make the following assumptions:

\begin{enumerate}
    \item We will assume that time is slotted into equal length of time slots.
          In the subsequent discussion, thelength of the time slot will be denoted by $T_s$.
    \item We will let $N$ denote the number of hosts.
          We will assume the hosts are identical and packets arriveto each host following a Poisson process with rate $\lambda$(pkts/sec).
    \item Hosts can transmit only at slot boundaries.
    \item For a new packet, the node attempts transmission in the next slot boundary.
    \item If at a particular slot boundary there are more than one host ready to transmit, there will be a collision.When there is a collision the packet is not received by the receiver and must be re-transmitted.
    \item How a node determines when to re-transmit depends on the algorithm.
          We will consider differentalgorithms as discussed in the next section.
    \item We will be plotting the throughput as a function of the arrival rate $\lambda$.
          Throughput is defined as the number of successful transmission per time unit.
          In the simulation, you can count the number of slots in which there is successful transmission and divided that by the total number of slots that you simulate.
\end{enumerate}

\section{Algorithms}


We will consider the following algorithms:

\begin{enumerate}
    \item p-Persistent ALOHA: In this case each active node re-transmits in a slot with probability $p$.
          We will consider two different values of $p$, 1)$p= 0.5$ and 2)$p=1/N$.
    \item Binary Exponential Backoff: When hosts collide, they will schedule their re-transmission usingthe following binary exponential backoff algorithm.
          The number of slots to delay after thenthre-transmission attempt is chosen as a uniformly distributed integer in the range $0 \le r \le 2^K$, where $K= min(n,10)$.
    \item Linear Backoff: When hosts collide, they will schedule their re-transmission using the following linear backoff algorithm.
          The number of slots to delay after thenthre-transmission attempt is chosenas a uniformly distributed integer in the range $0 \le r \le K$, where $K= min(n,1024)$.
\end{enumerate}

\section{Simulation Analysis}
Extend the simulation model of the single server queue that is given in the Jupyter Notebook to model the above system.
Based on the simulation model, obtain the following results:

Plot  the  throughput  as  a  function  of $\lambda$ with  the p−Persistent  Aloha,  Binary  Exponential  Backoff algorithm as described above.
Slot time $T_s= 1$ and number of hosts $N = 30$.
Obtain the throughput for the values of $\lambda = 0.01,0.02,0.03,0.04,0.05,0.06,0.07,0.08,0.09$.

\subsection*{Solution}

Note that the simulation code is rather long and I cannot post it here.
If you would like to examine the code and raw data, please contact me.

\begin{figure}[h]
    \centering
    \begin{tikzpicture}
        \begin{axis}[
                title={Throughput for p-Persistent($p = 1/30$), Binary Exponential Backoff and Linear Backoff},
                xmin=0, xmax=0.1,
                xtick={0, 0.02, ..., 0.12},
                ymin=0, ymax=0.6,
                ytick={0, 0.1, ..., 0.7},
                xlabel={$\lambda$},
                ylabel={Throughput},
                legend pos=south east,
                ymajorgrids=true,
                grid style=dashed,
            ]
            \addplot[red] table [x=lambda, y=throughput, col sep=comma] {plots/PPersistentBackOff(p=0.033).csv};
            \addplot[blue] table [x=lambda, y=throughput, col sep=comma] {plots/LinearBackOff.csv};
            \addplot[green] table [x=lambda, y=throughput, col sep=comma] {plots/BinaryExpBackOff.csv};
            \legend{p-Persistent($p = 1/30$), Binary Exponential, Linear Backoff}
        \end{axis}
    \end{tikzpicture}
\end{figure}


\begin{figure}[h]
    \centering
    \begin{tikzpicture}
        \begin{axis}[
                title={Throughput when p-Persistent($p = .5$) is used},
                xmin=0, xmax=0.1,
                xtick={0, 0.02, ..., 0.12},
                ymode=log,
                xlabel={$\lambda$},
                ylabel={Throughput},
                legend pos=north east,
                ymajorgrids=true,
                grid style=dashed,
            ]
            \addplot[red] table [x=lambda, y=throughput, col sep=comma] {plots/PPersistentBackOff(p=0.500).csv};
            \legend{p-Persistent($p = .5$)}
        \end{axis}
    \end{tikzpicture}
\end{figure}
\end{document}
\grid
